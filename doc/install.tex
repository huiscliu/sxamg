
\chapter{Installation}


\verb|SXAMG| uses {\color{blue}{\nverb{autoconf}}} and {\color{blue}{\nverb{make}}} 
to detect system parameters, to set default parameters, to build and to install.

\section{Configuration}
The simplest way to configure is to run command:
\begin{evb}
./configure
\end{evb}

This command will try to find optional packages if applicable and set system parameters.

\section{Options}
The script {\color{blue}{configure}} has many options, if user would like to check, 
run command:
\begin{evb}
./configure --help
\end{evb}

Output will be like this,
\begin{evb}
`configure' configures this package to adapt to many kinds of systems.

Usage: ./configure [OPTION]... [VAR=VALUE]...

To assign environment variables (e.g., CC, CFLAGS...), specify them as
VAR=VALUE.  See below for descriptions of some of the useful variables.

....

Optional Features:
  --disable-option-checking  ignore unrecognized --enable/--with options
  --disable-FEATURE       do not include FEATURE (same as --enable-FEATURE=no)
  --enable-FEATURE[=ARG]  include FEATURE [ARG=yes]
  --enable-rpath          enable use of rpath (default)
  --disable-rpath         disable use of rpath
  --with-rpath-flag=FLAG  compiler flag for rpath (e.g., "-Wl,-rpath,")
  --disable-assert        turn off assertions
  --enable-big-int        use long int for INT
  --disable-big-int       use int for INT (default),
  --with-int=type         integer type(long|long long)
  --enable-long-double    use long double for FLOAT
  --disable-long-double   use double for FLOAT (default)

Some influential environment variables:
  CC          C compiler command
  CFLAGS      C compiler flags
  LDFLAGS     linker flags, e.g. -L<lib dir> if you have libraries in a
              nonstandard directory <lib dir>
  LIBS        libraries to pass to the linker, e.g. -l<library>
  CPPFLAGS    (Objective) C/C++ preprocessor flags, e.g. -I<include dir> if
              you have headers in a nonstandard directory <include dir>
  CXX         C++ compiler command
  CXXFLAGS    C++ compiler flags
  FC          Fortran compiler command
  FCFLAGS     Fortran compiler flags
  CPP         C preprocessor
  CXXCPP      C++ preprocessor
\end{evb}

The most important options are,
\begin{itemize}
    \item \nverb{--prefix=PATH} where to install the library, and default directory is \verb|/usr/local/sxamg/|;

    \item \nverb{--enable-rpath} and \nverb{--disable-rpath}, use rpath or not, and it is enabled by default;

    \item \nverb{--enable-big-int} and \nverb{--disable-big-int}, use big integer or not, and use \nverb{int} by default;

    \item \nverb{--with-int=type}, type is \nverb{long} or \nverb{long long}. This option is checked when big integer
        is enabled (\nverb{--enable-big-int});

    \item \nverb{--enable-long-double} and \nverb{--disable-long-double}, use \nverb{long double} or not, and
        \nverb{double} is used by default;

\end{itemize}

\section{Compilation}
After configuration, \nverb{Makefile} and related scripts will be set correctly.
A simple {\color{blue}\textbf{make}} command can compile the package,

\begin{evb}
make
\end{evb}

A library, \verb|libsxamg.a|, will be generated under, \verb|lib/|.

\section{Installation}
Run command:
\begin{evb}
make install
\end{evb}
The package will be installed to a user-defined directory by \nverb{--prefix=DIR} during configuring, 
such as \nverb{--prefix=/usr/sxamg/}.
The default destination is \nverb{/usr/local/sxamg/}. 

